d) Demuestre que la probabilidad de hallar el momentum $p_1$ entre $p_1$ y $p_1+d p_1$ es proporcional
$$
P\left(p_1\right) \Delta p_1=\frac{\Omega_{\text {anillo }}}{\Omega_{\text {cascaron }}} \propto \frac{R^{3 N-3}}{R^{3 N-2}}=\frac{1}{R}\left[1-\frac{p_1^2}{2 m E}\right]^{\frac{3 N-3}{2}} \approx \frac{1}{R}[1-\epsilon]^{\frac{3 N}{2}} \quad, \text { con } \quad \epsilon=\frac{p_1^2}{2 m E} .
$$
Como el término entre corchetes cuadrados está elevado a un exponente enorme, la probabilidad será diferente de cero solamente si este término es similar a 1 , es decir, si $p_1 \sim 0$.