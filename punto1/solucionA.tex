\begin{answer}[literal a]
    Como el espacio de fase $6N$ dimensional dimensional, con $N$ el numero de particulas, se puede considerar como el producto cartesiano del espacio de configuraciones $3N$ dimensional
    y el espacio de momentos $3N$ dimensional, lo cual implica que es valido considerar por separado el volumen espacio de configuraciones y el volumen del espacio de momentos, de fomra 
    tal que el volumen total del espacio de fase es el producto de estos dos volumenes.\\
    El volumen del espacio de configuraciones es el volumen de un $N$-cubo de lado $L^N$, es decir $V^N$. Esto es asi debido a que un gas ocupa todo el espacion que lo contiene y en el espacion 
    de configuraciones por ejemplo para $1$ particula, tendremos que su volumen es $V$ por lo que para $N$ particulas el volumen sera $V^N$.\\
    El volumen del espacio de momentos es el volumen de una $N$-esfera de radio $\sqrt{2mE}$ el cual corresponde a:
    \begin{align*}
        V_{momentos} &= \frac{\pi^{3N/2}}{\Gamma(3N/2 + 1)} (\sqrt{2mE})^{3N} \\
        &= \frac{\pi^{3N/2}}{\Gamma(3N/2 + 1)} (2mE)^{3N/2} \\
    \end{align*}
    Pues los momentos de las particuals estan bajo la restriccion:
    $$\sum_{i=1}^{3N} p_i^2 = \sqrt{2mE} \qquad (1.a.1)$$
    con cada molecual teniendo $3$ grados de libertad.\\
    De esta forma el volumen del espacio de fase para los estado con energia menor igual que $E$ es:
    \begin{align*}
        V_{T}=V_{configuraciones} V_{momentos} \\
        &= V^N \frac{\pi^{3N/2}}{\Gamma(3N/2 + 1)} (2mE)^{3N/2} \\
        &= V^N \frac{\pi^{3N/2}}{\Gamma(3N/2 + 1)} (2m)^{3N/2} E^{3N/2} \\
    \end{align*}
\end{answer}