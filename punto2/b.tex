b) ¿Bajo qué condiciones se maximiza la probabilidad $P_I(k)$ ? Como el denominador $\Omega_N(n)$ es constante y el logaritmo es una función monótona, el máximo se obtiene cuando la entropía total $S(k)=S_I(k)+S_D(l)$ es máxima, $\operatorname{con} S_I(k)=\ln \Omega_I(k)$ y $S_D(l)=\ln \Omega_D(l)$. Derive la entropía total respecto a $k$, iguálela a cero y compruebe que,
$$
\frac{\partial S_I}{\partial k}=\frac{\partial S_D}{\partial l},
$$
que es para nuestro caso la versión de que la entropía se maximiza cuando las "temperaturas" $T_I=\frac{\partial S_I}{\partial k}$ y $T_D=\frac{\partial S_D}{\partial k}$ son iguales.