Imagine un gas ideal de $n$ moléculas idénticas que se encuentra en una caja de $N$ celdas que podemos dividir mentalmente en dos zonas: una zona izquierda (I) de $I$ celdas, y otra derecha (D) de $D$ celdas. Asumamos que cada celda puede tener a lo más una molécula. Si todas las formas de repartir las $n$ moléculas en las $N$ celdas son igualmente probables, ¿por qué es casi imposible ver que todas las moléculas se coloquen en una de las dos zonas, dejando a la otra vacía? Vamos a estudiar el problema de dos maneras: analítica y computacional.
Para la parte analítica,

