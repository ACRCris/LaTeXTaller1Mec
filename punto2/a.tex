a) Estime la probabilidad $P_l(k)$ de encontrar $k$ moléculas a la izquierda contando todas las maneras posibles de colocar $k$ moléculas en las cajas de la zona I (y, por lo tanto, $l=n-k$ moléculas en la zona $\mathrm{D}$ ) y dividiéndola por el número de maneras posibles de colocar las n moléculas en todas las $N=\mathrm{I}+\mathrm{D}$ cajas. Compruebe que la razón resulta ser,
$$
P_I(k)=\frac{\Omega_I(k) \Omega_D(l)}{\Omega_N(n)},
$$
con
$$
\Omega_l(k)=\left(\begin{array}{l}
I \\
k
\end{array}\right), \quad \Omega_l(k)=\left(\begin{array}{l}
D \\
l
\end{array}\right), \Omega_N(n)=\left(\begin{array}{l}
N \\
n
\end{array}\right)
$$