\begin{answer}
    Si vemos un ensamble estadistico como $P\subseteq \Omega \times [0,1]$, es decir,
    \begin{align*}
        P: &\Omega \mapsto [0,1]\\
        &\omega \mapsto P(\omega) \in [0,1]
    \end{align*}
    tal que $P := \{(\omega,\rho): \omega \in \Omega \quad \text{y} \quad \rho \in [0,1]\}$ si $\Omega$ es finito y sea $\{p_i\}_{i=1}^n$ un familia de conjuntlo de $P$ tal que $p_i \cap p_ j = \emptyset$ para $i\neq j$ y ${\times}_{i=1}^n p_i = P$ entonces la entropia 
    del ensamble la cual se define como:
    \begin{align*}
        S(P) &= -\sum_{(\omega,\rho) \in P} \rho \log \rho = -\sum_{i = 1}^{|\Omega|} P(\omega_i) \log P(\omega_i)\\
        &= -\sum_{i = 1}^{|\Omega|} \rho_i \log \rho_i
    \end{align*} 
    \begin{enumerate}
        \item $S(P) = S\left(\times_{i=1}^n p_i\right) = \sum_{i=1}^n S(p_i)$
        \item para $i\neq j$ $S(p_i \times p_j) = S(p_i) + S(p_j)$
    \end{enumerate}
    Para un gas ideal de $n$ moleculas en un volumen $V$ particionado en $N$ celda $V_i$ con $i = 1,2,\dots,N$ y $V = \sum_{i=1}^N V_i$ entonces el cojunto de casos totales es 
    \begin{align*}
        \Omega = \{b_0b_1\dots b_N: b_i \in \{0,1\}\} = \{k: k = b_i 2^i \quad \text{con} \quad b_i \in \{0,1\}\}
    \end{align*}
    con:
    $$
     b_i = \begin{cases}
        1 & \text{si la molecula esta en el celda } V_i\\
        0 & \text{en otro caso}
    \end{cases}
    $$
    Si dividimos le cadena de bits en los primeros $I$ bits con $k$ unos y los ultimos $D = N - I$ bits con $l= n-k$ entoces el numero de formas en las que puedo colocar los $k$ unos en los primeros $I$ bits es:
    \begin{align*}
        |\Omega_I(K)|= \binom{I}{k} = \frac{I!}{k!(I-k)!}
    \end{align*}
    donde:
    \begin{align*}
        \Omega_I(K) &= \{b_0b_1 \dots b_I: b_i \in \{0,1\}\}\\
        \Omega_D(K) &= \{b_{0}b_{1}\dots b_D: b_i \in \{0,1\}\}\\
    \end{align*}
    entonces el ensamble microcanonico es para $\Omega_I(K)$:
    \begin{align*}
        P_I := \left\{ \left(\omega, \rho \right): \omega \in \Omega_I(K)  \quad \text{y} \quad = \frac{1}{|\Omega_I(K)|}\right\}
    \end{align*}
    De la forma analoga definimos el ensamble microcanonico para $\Omega_D(L)$:
    \begin{align*}
        P_D := \left\{ \left(\omega, \rho \right): \omega \in \Omega_D(L)  \quad \text{y} \quad = \frac{1}{|\Omega_D(L)|}\right\}
    \end{align*}
    Veamos que $P_I \cap P_D = \emptyset$ y $P_I \times P_D = P$ entonces
    \begin{align*}
        S(P_k) &= S(P_I \times P_D) = S(P_I) + S(P_D)\\
        &= S(P_I) + S(P_D)\\
        &= \log |\Omega_I(K)| + \log |\Omega_D(L)|\\
    \end{align*}
    Por lo tanto la entropia total $S(k) = S_l(k) + S_D(l)$ es maxima cuando:
    \begin{align*}
        \frac{d S(k)}{d k} = \frac{d S_l(k)}{d k} + \frac{d S_D(l)} {d l} \frac {d l}{d k} = 0  &\quad \Rightarrow \quad \frac{d S_l(k)}{d k} - \frac{d S_D(l)} {d l} = 0\\
        &\quad \Rightarrow \quad \frac{d S_l(k)}{d k} = \frac{d S_D(l)} {d l}
    \end{align*}
\end{answer}