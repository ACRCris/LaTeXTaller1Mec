\begin{answer}
    Si vemos un ensamble estadistico como $P\subseteq \Omega \times [0,1]$, es decir,
    \begin{align*}
        P: &\Omega \mapsto [0,1]\\
        &\omega \mapsto P(\omega) \in [0,1]
    \end{align*}
    tal que $P := \{(\omega,\rho): \omega \in \Omega \quad \text{y} \quad \rho \in [0,1]\}$ si $\Omega$ es finito y sea $\{p_i\}_{i=1}^n$ un familia de subconjunto de $P$ tal que $p_i \cap p_ j = \emptyset$ para $i\neq j$ y $\bigcup_{i=1}^n p_i = P$ entonces la entropia 
    del ensamble la cual se define como:
    \begin{align*}
        S(P) &= -\sum_{(\omega,\rho) \in P} \rho \log \rho = -\sum_{i = 1}^{|\Omega|} P(\omega_i) \log P(\omega_i)\\
        &= -\sum_{i = 1}^{|\Omega|} \rho_i \log \rho_i
    \end{align*} 
    \begin{enumerate}
        \item $S(P) = S\left(\bigcup_{i=1}^n p_i\right) = \sum_{i=1}^n S(p_i)$
        \item para $i\neq j$ $S(p_i \cup p_j) = S(p_i) + S(p_j)$
    \end{enumerate}
    Para un gas ideal de $n$ moleculas en un volumen $V$ particionado en $N$ celda $V_i$ con $i = 1,2,\dots,N$ y $V = \sum_{i=1}^N V_i$ entonces el cojunto de casos totales es 
    \begin{align*}
        \Omega = \{b_0b_1\dots b_N: b_i \in \{0,1\}\} = \{k: k = b_i 2^i \quad \text{con} \quad b_i \in \{0,1\}\}
    \end{align*}
    con:
    $$
     b_i = \begin{cases}
        1 & \text{si la molecula esta en el celda } V_i\\
        0 & \text{en otro caso}
    \end{cases}
    $$ 
    Como solo hay $n$ moleculas es decir $n$-unos entonces $|\Omega| = \binom{N}{n}$ y como las cadenas de bits son iguales solo si representan el mismo estado entonces para todo entoces la probabilidad de cada estado es $P(\omega) = \frac{1}{|\Omega|}$ y por lo tanto 
    tenemos el siguiente ensamble microcanonico:
    \begin{align*} 
    P = \left\{(\omega,\rho): \omega \in \Omega \quad \text{y} \quad \rho = \frac{1}{|\Omega|}\right\}
    \end{align*}
    Y definamos:
    \begin{align*}
        \Omega_I := \{b_0b_1\dots b_I0\dots 0: b_i \in \{0,1\}\} =  
    \end{align*}
    \begin{align*}
        \Omega_I := \{0\dots 0b_0b_1\dots b_D: b_i \in \{0,1\}\} =  
    \end{align*}
    \begin{align*}
        P_I = \left\{(\omega,\rho): \omega \in \Omega_I \quad  \text{ y} \quad \rho = \frac{1}{|\Omega|}\right\}
    \end{align*}

\end{answer}